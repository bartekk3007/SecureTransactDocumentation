% słaba funkcja
% \titlecontents{chapter}[0cm]{}{\normalsize\bfseries{\appendixname\space}\contentslabel{-1pt}\hspace*{0.6cm}}{}{\titlerule*[4.5pt]{.}\hspace*{5pt}\contentspage}

\chapter*{Dodatek A: Uzupełnienie wymogów}
\addcontentsline{toc}{chapter}{Dodatek A}
% Póki co jest problem ze słowem dodatek w spisie treści. Funkcja wklejona na początku działa słabo. Powyższe linijki obchodzą ten problem bardzo dobrze, ale dodatki trzeba sobie numerować "ręcznie"

No i mamy jeszcze dodatek - jeśli trzeba

Dodatki należy oznaczać kolejnymi dużymi literami alfabetu. W dodatkach należy umieszczać elementy uzupełniające, które powinny zostać dołączone do pracy, np. w celu prezentacji wykonanych obliczeń, schematy ideowe.