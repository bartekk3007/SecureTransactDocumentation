\chapter{Super rozdział}
\label{chap:super}

Mam tu jeszcze raz tabelę \ang{table}, tym razem to tabela \ref{tab:interakcja}. Poprzednią tabelą była tabela \ref{tab:font_size} - \zobrozdz{chap:wstep}

\begin{table}[!ht]

\caption{Możliwości obliczeniowe współczesnych systemów}
\label{tab:interakcja}
\small
\begin{tabular}{|p{0.47\textwidth}|p{0.47\textwidth}|}
\hline

%-
\begin{center}
\textbf{Wnioski z~hipotezy Turinga}
\end{center}
 & 
\begin{center}
\textbf{Wnioski zaproponowane przez~Wegnera i~Goldin}
\end{center}

\\\hline 

%-
  Wszystkie problemy \emph{obliczalne} można rozwiązać za~pomocą funkcji. &  Wszystkie problemy \emph{algorytmiczne} można rozwiązać za~pomocą funkcji. \\\hline
%-
	Wszystkie problemy \emph{obliczalne} można opisać algorytmem. &  Wszystkie problemy \emph{oparte na~funkcjach} można opisać algorytmem.\\\hline
%-
	Algorytmy są tym, co~mogą wykonywać \emph{dowolne komputery}. & Algorytmy są tym, co~mogły wykonywać \emph{wczesne} komputery. \\\hline
%-
	Maszyna Turinga jest ogólnym modelem \emph{dowolnych} komputerów.  &  Maszyna Turinga jest ogólnym modelem \emph{wczesnych} komputerów.\\\hline
%-
	Maszyna Turinga potrafi zasymulować \emph{dowolny} komputer.  &  Maszyna Turinga potrafi zasymulować \emph{dowolne urządzenie} realizujące algorytmiczne obliczenia.\\\hline
%-
	
	\begin{center}
	\Large{\sffamily{:(}}
	\end{center}
			 &  Maszyna Turinga nie~potrafi rozwiązać każdego problemu obliczeniowego ani wykonać wszystkich operacji, które realizują \emph{współczesne systemy}.\\\hline


\end{tabular}
\end{table}


I może jeszcze logo:

\begin{figure}[!ht]
	\centering
		\includegraphics[width=0.3\textwidth]{Images/logo_pg.jpg}
	\caption{Oto logo PG}
	\label{fig:PD}
\end{figure}


