\chapter{Dokument elektroniczny}
\label{chap:DokumentElektroniczny}
Dokument, najogólniej, to nośnik zrozumiałej dla~człowieka treści, zwanej potocznie informacją. Stanowi naturalny, utrwalony przez~wieki interfejs dla~człowieka oraz powszechny środek komunikacji międzyludzkiej. Jako dokument mogą być rozumiane wszelkie środki przekazu informacji, od~tradycyjnych dokumentów tekstowo-obrazkowych, odbieranych za~pośrednictwem wzroku, po~coraz bardziej powszechne dokumenty multimedialne, szczególnie dźwiękowe, ale też odbierane poprzez dotyk (np. napisane alfabetem Braille'a) bądź nawet przez~węch\footnote{Popularny był kiedyś zwyczaj perfumowania listów, co~pozwalało identyfikować jego nadawcę.}. 

\dots

%---
\section{Notacje reprezentacji informacji}
\label{sec:NotacjeReprezentacjiInformacji}

\emph{Format dokumentu}, jako ustalony standard zapisu informacji definiujący jego strukturę i~zawartość, jest jedną z~głównych cech rozróżniających dokumenty \cite{techterms}. Istnieje wiele przyczyn występowania różnych formatów. Najczęściej są to względy komercyjne, tj. wspieranie pewnych formatów przez~firmy informatyczne lub uzależnienie rodzaju zapisu danych od~określonego oprogramowania dostępnego na~rynku. Często w~wyborze formatu istotną rolę pełni jego łatwość użycia, siła wyrazu, uniwersalność, sposób kompresji oraz zrozumiałość dla~użytkowników. Formaty można, w~ogólności, podzielić na~otwarte, gdy opis standardu jest powszechnie dostępny oraz zamknięte, gdy szczegóły standardu znane są tylko producentowi. 

\emph{Typ dokumentu} jest pojęciem niezwiązanym bezpośrednio z~technicznym opisem struktury dokumentu. Jest to pewna nazwa nadana konkretnym rodzajom dokumentów w~celu powiązania ich z~aplikacjami, które mogą je prawidłowo odczytać \cite{techterms}. Ze względu na~to, że istnieje bardzo wiele typów dokumentów, ich właściwa identyfikacja jest kluczowa w~przypadku konieczności przenoszenia dokumentów między różnymi urządzeniami. W~takiej sytuacji pomocne są ogólnodostępne repozytoria, które zapewniają jednolity opis typów dokumentów dla~wielu ich odbiorców.

%---
\subsection{Repozytorium typów dokumentów}
\label{sec:RepozytoriumTypowDokumentow}

Dominującym obecnie mechanizmem wymiany dokumentów różnych typów jest poczta elektroniczna. Dokument przesyłany jako wiadomość elektroniczna (email) może zawierać w~sobie dokumenty dowolnego typu w~postaci zarówno treści wiadomości jak i~załączników. Mimo że wiadomość elektroniczna jest przekształcana na~ciąg znaków możliwych do~przesłania przez~protokoły pocztowe \cite{SMTP_RFC}, agenty pocztowe są w~stanie odebrać i~odpowiednio odtworzyć zawartość wiadomości. Jest to możliwe dzięki standardowi MIME \ang{Multipurpose Internet Mail Extensions}  \cite{mime-format}, powszechnie stosowanemu przy przesyłaniu poczty elektronicznej.

\dots

%---
\subsection{Notacje dokumentów a~praca grupowa}
\label{sec:notacjaAPracaGrupowa}

Z jednej strony postać cyfrowa ułatwia przekazywanie dokumentów, z~drugiej różnorodność notacji ogranicza możliwości ich prawidłowego odczytu i~edycji.
Typy zawartości MIME wspierają pracę grupową opartą na~dokumencie i~pozwalają zidentyfikować typ niemal każdego dokumentu napisanego przez~człowieka, co~umożliwia jego przesyłanie z~wykorzystaniem protokołów poczty elektronicznej oraz powiązanie go z~odpowiednią aplikacją. Nie rozwiązuje to jednak dwóch problemów związanych z~przekazywaniem dokumentów. Oba dotyczą właściwego wyboru aplikacji do~odpowiedniego przetwarzania nietekstowych zawartości:

\begin{enumerate}
	\item szczegóły odtworzenia otrzymanej zawartości opisanej znanym typem MIME mogą wymagać wcześniejszych uzgodnień z~nadawcą wiadomości,
	\item niestandardowa zawartość wiadomości może wymagać od~odbiorcy komunikatu odnalezienia i~zainstalowania odpowiedniej aplikacji.
	\label{mark:problemyZawartosci}
\end{enumerate}

Sytuacja 1. zazwyczaj nie~ma miejsca, gdy praca nad~dokumentem odbywa się w~grupie, która posiada takie samo oprogramowanie do~edycji zawartości danego typu (tego samego producenta i~tej samej wersji). Jednak jest to sytuacja rzadka w~realiach organizacji wirtualnych, których członkowie znajdują się w~terytorialnym rozproszeniu, zmieniają swoje lokalizacje i~urządzenia, na~których pracują (wykonują, przykładowo, część pracy w~biurze a~część w~domu).

\dots



%---
\subsection{Klasyfikacja typów dokumentów}
\label{sec:KlasyfikacjaTypowDokumentow}

Typy dokumentów można sklasyfikować w~sposób przedstawiony na~rysunku \ref{fig:docFormatsPL}. Podział ten dotyczy możliwości przetwarzania zawartości dokumentów przez~urządzenia elektroniczne. Ogólnie, segreguje dokumenty począwszy od~zrozumiałych tylko przez~ludzi do~zrozumiałych jednocześnie przez~ludzi i~urządzenia elektroniczne\footnote{Zrozumiałość, w~tym przypadku, polega na~możliwości automatycznej analizy treści przez~program komputerowy.}. 

\begin{figure}[htbp]
	%\centering
 	  \resizebox{.88\textwidth}{!}{\input{Images/docFormatsPL.latex}}
	\caption{Klasyfikacja typów dokumentów}
	\label{fig:docFormatsPL}
\end{figure}

\subsubsection{Dokumenty analogowe i~cyfrowe}
\label{sec:DokumentyAnalogoweICyfrowe}

Dokumenty można podzielić ze~względu na~nośnik na~dwie główne grupy: analogowe i~cyfrowe. Jako \emph{dokumenty analogowe} określone zostały wszystkie dokumenty będące w~formie niezdygitalizowanej, czego przykładem są dokumenty papierowe (drukowane lub pisane odręcznie). Mogą to być także dowolne zasoby informacji, takie jak płyty winylowe, zielniki, albumy ze~zdjęciami itp. obiekty informacyjne. Wiele organizacji, np. sądowniczych, wymaga obecności dokumentów analogowych ze~względów formalnych, prawnych lub dlatego, że niosą one poza treścią dodatkowe informacje. Występuje również wiele historycznych dokumentów analogowych, istotnych dla~różnych organizacji. Stąd zasadność umieszczenia dokumentów analogowych w~klasyfikacji typów dokumentów.

Gdy dany dokument analogowy zostanie przetworzony na~zrozumiałą dla~urządzenia elektronicznego postać numeryczną, wówczas możemy mówić o~\emph{dokumencie cyfrowym}. Są także dokumenty cyfrowe typu „born electronic”, które powstały w~środowisku komputerowym od~razu jako cyfrowe i~nie mają swoich analogowych poprzedników.

Istnieje wiele formatów dokumentów cyfrowych, które mają różne możliwości i~zastosowania. Zasadniczo można podzielić je odnośnie reprezentacji na~dwie klasy: binarne i~tekstowe, gdzie te pierwsze są nieprzetworzonym zapisem dokumentu analogowego w~postaci zestawu próbek informacji, np. pikseli, zaś te drugie są pewnym zestawem rozpoznawalnych symboli, które mogą mieć różną złożoność i~strukturę.

\subsubsection{Dokumenty binarne} 
\label{sec:DokumentyBinarne}

Dokumenty binarne będące przeniesieniem dokumentów analogowych (papierowych) do~środowiska komputerowego są reprezentowane jako zestaw pikseli o~określonych cechach. Przykładowe formaty takiego zapisu to JPG, BMP czy TIFF. 
Treść tego typu dokumentów jest w~praktyce trudna do~rozpoznania przez~komputer ze~względu na~obecność szumów obniżających jego czytelność. 
Jednak taki dokument ma także zalety: może być przechowywany w~pamięci komputera i~przesyłany przez~sieć, a~po~wyrenderowaniu na~ekran lub wydruku na~papierze może być przez~człowieka traktowany jak oryginał. Nawet takie artefakty, jak plamy, zagniecenia czy tekstura papieru mogą nieść ważną informację dla~jego użytkownika.

Ogólnie, analiza treści dokumentów w~postaci binarnej wymaga wsparcia zewnętrznej aplikacji i~jest związana z~cyfrowym przetwarzaniem obrazów. Jest to problem złożony i~wciąż otwarty, rozwiązywany na~wiele sposobów metodami sztucznej inteligencji. Metody te są coraz bardziej skuteczne, jednakże nie~pozbawione błędów -- zwłaszcza w~przypadku tekstów odręcznych lub mocno zaszumionych \cite{OCRReview, Handwritting:2011}. Dużo lepiej wygląda natomiast kwestia wizualizacji dokumentów binarnych i~wiąże się po~prostu z~wyświetleniem pikseli. Oczywiście, pojawiają się tu problemy rozdzielczości, kompresji czy częstości próbkowania sygnału, ale są to kwestie obecnie dość dobrze rozwiązywane, ze~względu na~coraz lepsze możliwości sprzętu. 

\subsubsection{Dokumenty tekstowe}
\label{sec:DokumentyTekstowe}

Dokumenty tekstowe zawierają wyłącznie ciągi symboli rozpoznawalne bezpośrednio przez~komputer i~człowieka. Ich podział uzależniony jest od~możliwości przetwarzania zawartych w~nich informacji. Można wyróżnić dokumenty parsowalne i~nieparsowalne. 

\emph{Nieparsowalne} dokumenty tekstowe zawierają strumień znaków wyrażony w~języku naturalnym, np. bezpośredni zapis rozmowy telefonicznej w~formie stenogramu lub tekstu. Taka zawartość dokumentu jest oznaczona przez~typ MIME: \texttt{text/plain}. Jedynym sposobem automatycznego wydobywania potrzebnych informacji z~takiego dokumentu jest użycie narzędzi do~przetwarzaniu języka naturalnego. Wizualizacja zaś przebiega na~zasadzie kopiowania i~wyświetlania strumienia tekstowego i~jest całkowicie zależna od~sposobu prezentowania go przez~odpowiedni program do~tego służący.

Grupą dokumentów znacznie łatwiej przetwarzanych przez~komputer są dokumenty \emph{parsowalne}, zwane również \emph{dokumentami elektronicznymi}. Są to dokumenty posiadające oprócz treści, także strukturę, która jest zrozumiała zarówno przez~człowieka jak i~przez komputer. W~celu zdefiniowania dokumentu elektronicznego można więc użyć następującej równości:

\begin{center}
\emph{dokument elektroniczny = struktura + zawartość.
}
\end{center}

Ze względu na~składnię, dokumenty elektroniczne można podzielić na: znakowane i~formalne. \emph{Formalne} dokumenty elektroniczne zawierają kod źródłowy programów komputerowych, a~ich gramatyka i~składnia są ściśle sprecyzowane. W~wiadomościach MIME są identyfikowane przez~odpowiednie podtypy, przykładowo \texttt{text/x-java} dla~parsowalnego formalnego kodu źródłowego w~języku Java. Wydobywanie treści odbywa się poprzez rozbiór dokumentu przez~odpowiednie analizatory składni (parsery) i~jest w~pełni skuteczne (np. narzędzie do~generowania dokumentacji Javadoc \cite{JAVADOC}). Dobrze zdefiniowana gramatyka określa jednoznacznie elementy składni: zmienne, deklaracje typów, instrukcje itp. Wyświetlanie dokumentów formalnych odbywa się również przez~analizę składni i~jest dokonywane przez~narzędzia służące do~prezentacji kodu źródłowego, np. edytory posiadające wsparcie dla~danego języka i~umożliwiające automatyczne formatowanie wydruku \ang{pretty printing}.

Dokumenty \emph{znakowane} zawierają w~swojej treści fragmenty opisane odpowiednimi znacznikami. Znaczniki te organizują zawartość dokumentu, służąc jego prezentacji lub nadając mu strukturę logiczną. Gdy znaczniki wprowadzane są w~celu zarządzania wyglądem dokumentu, tak jak w~dokumentach HTML, wówczas ich wizualizacja jest bardzo prosta i~przebiega „w locie” (jest interpretowana w~trakcie czytania dokumentu przez~odpowiednią aplikację). Wydobywanie treści zaś odbywa się poprzez transformację -- jest trudniejsze i~mniej jednoznaczne niż w~przypadku dokumentów posiadających strukturę logiczną, jak np. XML. W~dokumentach ze~strukturą logiczną znaczniki wprowadzane są w~celu opisania zawartości, a~nie wyglądu dokumentu, toteż wydobywanie treści odbywa się poprzez wyszukanie odpowiednich znaczników, co~jest jednoznaczne i~natychmiastowe. Wizualizacja wymaga zaś transformacji dokumentu do~oczekiwanego formatu, np. w~XML żądany wygląd można uzyskać za~pomocą tzw. obiektów formatujących \ang{formatting object}, np. XSL-FO \cite{XSL-FO}.

Zaprezentowana klasyfikacja typów dokumentów ukazuje różnorodność notacji informacji, a~co za~tym idzie złożoność problemu automatycznej analizy zawartości. Użytkownik w~swojej pracy spotyka się z~wieloma typami dokumentów, co~nierzadko sprawia mu problemy i~prowadzi do~przedłużania się pracy z~danym dokumentem. Zmienność istniejących formatów i~pojawianie się nowych rodzi problemy związane z~prawidłową identyfikacją i~przetwarzaniem zawartości. Dlatego rozwój dokumentów powinien iść w~kierunku ich kompatybilności oraz wsparcia użytkowników w~pracy grupowej z~różnorodną zawartością.

\section{Rozwój dokumentów elektronicznych}
\label{sec:RozwójDokumentówElektronicznych}

Powszechność wymienionych wcześniej problemów potencjalnej niekompatybilności znanej zawartości lub prawidłowego odtworzenia nieznanej zawartości nasuwa pytanie, czy jest możliwe stworzenie agenta zdolnego do~analizy treści dokumentu w~celu wsparcia użytkownika w~pracy nad~tym dokumentem. Rozwój dokumentów elektronicznych potwierdza potrzebę dostosowania ich struktury wewnętrznej do~analizy zawartości i~wspierania pracy grupowej. 
Użytkownicy oczekują od~dokumentów już nie~tylko możliwości wygodnego przeglądania treści, ale też tego, by dokument zachowywał wszystkie zalety dokumentów w~formie papierowej oraz oferował dodatkowe funkcje, w~tym szczególnie wspierał pracę grupową, aktywnie zabezpieczał dostęp użytkowników do~określonych fragmentów dokumentu, personalizował sposób wykorzystania zawartości, np. poprzez wyświetlanie jej w~języku użytkownika, dawał możliwości nanoszenia przypisów, wprowadzania odsyłaczy, itd. Pojawiające się rozwiązania dają możliwość łączenia różnorodnych typów dokumentów w~ramach jednej struktury oraz zmieniają rolę dokumentów w~procesie ich przetwarzania ze~statycznej w~aktywną. 

Rozwój dokumentów przyczynił się do~powstania dziedziny informatyki zwanej \emph{inżynierią dokumentu} \ang{document engineering} \cite{doc_centric2, doc_centric}. 

\dots

\subsection{Warstwowe dokumenty wielopostaciowe}
\label{sec:WarstwoweDokumentyWielopostaciowe}
Jednym z~pomysłów pozwalających na~łączenie różnych typów zawartości w~ramach jednej struktury oraz zarządzania poszczególnymi częściami dokumentu jest model dokumentu warstwowego zaproponowany przez~Phelpsa i~Wilensky'ego w~\cite{Phelps96, Phelps98_tech_rep} i~stanowiący punkt wyjściowy dla~moich badań nad~rozwojem dokumentów elektronicznych \cite{KKTI06_moja, TECHNICON06_moja}. Pozwala on stworzyć jeden dokument składający się z~kilku warstw o~różnej zawartości. Warstwy, dzięki odpowiednim znacznikom, tworzą zazwyczaj strukturę drzewa -- można je dowolnie ukrywać i~uwidaczniać, czyniąc ich treść opcjonalną. Taka struktura umożliwia nanoszenie przypisów, podkreśleń czy adnotacji, podobnie jak w~dokumentach papierowych. Umieszczane są one w~innych warstwach niż oryginalny dokument bazowy, co~pozwala chronić jego treść a~także ustalać politykę dostępu do~naniesionych treści. Innym przykładem zastosowania warstw jest możliwość „nałożenia” tłumaczenia na~istniejący już dokument zamiast tworzyć nowy dokument w~innym języku. Warstwy umożliwiają również bardziej zaawansowane współdzielenie dokumentów. Użytkownicy mogą ustalać prawa dostępu do~poszczególnych części dokumentu, np. zabezpieczać je hasłem. Pozwalają także na~tworzenie dokumentów rozproszonych, którego poszczególne warstwy znajdują się na~różnych urządzeniach elektronicznych i~są wyświetlane jako całość przez~odpowiednio do~tego przystosowaną aplikację.

Format PDF \ang{Portable Document Format} jest pierwszym formatem dokumentów, w~którym warstwowość została wprowadzona jako obowiązujący standard i~jest rozwijana na~dużą skalę. Począwszy od~wersji PDF 1.5 \cite{PDFRef} został wprowadzony mechanizm opcjonalnej zawartości \ang{optional content} i~dotyczy sekcji zawartości dokumentu, która może być dowolnie uwidaczniana i~ukrywana przez~autora dokumentu lub jego użytkownika. Sprawiło to, iż warstwy dokumentu stały się jeszcze bardziej praktyczne, gdyż istnieje możliwość dowolnej ich prezentacji, zarządzania pojedynczymi warstwami, podwarstwami a~także grupami warstw. Opcjonalna zawartość pozwala na~tworzenie warstw treści, rysunków, a~także wielojęzycznych dokumentów. Zawartość warstw jest łączona w~grupy OCG \ang{optional content group}, które są podstawową strukturą używaną do~sterowania widocznością składowych dokumentu. Reprezentują one pewne kolekcje elementów, które mogą być dynamicznie ukrywane bądź uwidaczniane przez~użytkownika. Elementy należące do~danej grupy mogą się znajdować w~różnych miejscach dokumentu i~należeć do~różnych strumieni zawartości. Grupie przypisywany jest stan ON (włączona) lub OFF (wyłączona). W~podstawowym przypadku zawartość należąca do~jednej grupy jest widoczna, gdy stan jest ON i~niewidoczna, gdy jest OFF, ale w~bardziej złożonych przypadkach zawartość może należeć do~kilku grup o~wykluczających się stanach. Wówczas można ustalić pewną taktykę widoczności, dla~zawartości należącej do~konkretnego zgrupowania danych. Przykładowo:

\begin{verbatim}
<</Type /OCMD                  %grupowanie danych w OCMD
  /OCGs[12 0 R 13 0 R 14 0 R]  %słownik składa się z trzech grup
                               %opcjonalnej zawartości
  /P /AllOn                    %stan P zostaje ustalony na AllOn, co
>>                             %oznacza, że zawartość jest widoczna, 
                               %jeśli stany każdej z trzech grup są ON,
                               %w przeciwnym przypadku jest ukryta           
\end{verbatim}

Phelps i~Wilensky \cite{Phelps96, Phelps98_tech_rep} rozszerzają model warstwowy dodatkowo o~funkcjonalność, proponując model \emph{dokumentu wielopostaciowego} (MVD, ang.\emph{ multivalent document}). Dokument składa się z~rozproszonej zawartości i~dynamicznych programów, zwanych odpowiednio \emph{warstwami} \ang{layers} i~\emph{zachowaniami} \ang{behaviors}. Warstwy, jak opisano wyżej, dają możliwość włączenia różnych typów zawartości w~strukturę jednego dokumentu. Mogą być też dodawane później do~utworzonego już dokumentu. Warstwy odzwierciedlają semantykę zawartości, i~nie wynikają z~lokalizacji w~dokumencie -- można powiedzieć, są to \emph{warstwy semantyczne}. 

\dots


\subsection{Dokumenty aktywne}
\label{sec:DokumentyAktywne}

Dodanie funkcjonalności do~dokumentu zmienia jego rolę w~procesie przetwarzania z~pasywnej na~aktywną. Możliwość wykorzystania dowolnego języka programowania do~specyfikacji zachowań dokumentu, czyni jego funkcjonalność otwartą i~potencjalnie nieograniczoną. Stąd można zdefiniować \emph{dokument aktywny} jako dokument elektroniczny zawierający wbudowany w~niego zbiór usług, definiujący jego funkcjonalność \cite{Ciancarini2002}, zatem: 

\begin{center}
\emph{dokument aktywny = struktura + zawartość + funkcjonalność}

\end{center}

Pierwsza koncepcja w~pełni aktywnych dokumentów elektronicznych zdolnych do~samodzielnego działania niezależnie od~aktualnej lokalizacji \ang{placeless documents} została zaproponowana przez~Dourish i~in. \cite{Dourish2000, Dourish2003}. Model ten wykorzystuje możliwość wbudowywania funkcjonalności w~dokument w~celu zarządzania jego zawartością. Głównym założeniem autorów było odejście od~standardowej hierarchicznej struktury i~ścisłej kategoryzacji dokumentów w~katalogach systemów operacyjnych, systemach mailowych czy na~stronach internetowych. Tradycyjnie bowiem umieszcza się dokument w~konkretnym katalogu znajdującym się w~pewnej statycznej hierarchii katalogów. Odszukanie takiego dokumentu może być utrudnione, zwłaszcza gdy zapisywany był zgodnie z~innym kryterium niż później odszukiwany lub jeśli ma być odszukany przez~inną osobę, która zupełnie tego kryterium nie~znała. Autorzy zauważyli, że katalogi plików nie~tylko zapewniają organizację dokumentów (grupowanie, lokalizację), ale także służą do~zarządzania nimi, np. tworzenia archiwów czy umożliwiania zdalnego dostępu. Aby skuteczniej zarządzać dokumentami zaproponowali więc odejście od~tradycyjnej hierarchii katalogów poprzez wprowadzenie własności \ang{properties} dokumentów, które pozwolą uniezależniać dokumenty od~ich lokalizacji i~dostosowywać je dynamicznie do~kontekstu użycia. 

\dots

\subsection{Dokumenty jako agenty}
\label{sec:DokumentyAgentowe}
Własności ,,wykonawcze" dokumentów \emph{Placeless documents} są uruchamiane po~zainicjowaniu akcji na~dokumencie traktowanym jako obiekt. Takie dokumenty można nazwać \emph{dokumentami reaktywnymi} \ang{reactive documents}, gdyż reagują samodzielnie na~każdą próbę dostępu do~nich. Alternatywą są \emph{dokumenty proaktywne} \ang{proactive documents}, które posiadają pewną autonomię i~same potrafią się uaktywnić lub dostosować do~środowiska, w~którym się aktualnie znajdują -- mają cechy autonomicznego programu komputerowego zwanego agentem \ang{software agent}. Tą klasę dokumentów będę nazywać \emph{dokumentami-agentami} \ang{document agents}\index{agent!dokument}, by podkreślić ich dwojaką naturę: klasycznie rozumianych dokumentów elektronicznych oraz agentów \cite{Ciancarini2002}.

W \cite{Satoh2001} została przedstawiona koncepcja dokumentów w~postaci mobilnych agentów

\dots