\chapter{Podsumowanie}
Udało się nam zrealizować wszystkie stawiane systemowi wymagania:
\begin{itemize}
    \item System działa w architekturze rozproszonej, każdy z użytkowników jest równy innym, system został zaimplementowany bez użycia głównego serwera, co pozwoliło uniknąć centralizacji
    \item W systemie można wysyłać wiadomości do innych użytkowników
    \item W systemie można tworzyć grupy z dowolnymi użytkownikami i treść wiadomości dla osób z poza grupy nie jest dostępna
    \item System oparty jest o mechanizmy działające w kryptowalutach mające na celu zapewnienie w nim uczciwości i bezpieczestwa
    \item System posiada graficzny interfejs użytkownika, pozwalający na łatwiejsze korzystanie z niego
    \item System posiada szyfrowanie danych w celu zapewnienia poufności i tego, aby nikt z poza sieci nie mógł odczytać treści wiadmości
    \item W systemie zapewniona jest niezaprzeczalność wiadomości poprzez ich podpisywanie podpisem cyfrowym przez weryfiktora
    \item W systemie zaimplementowane są mechanizmy zapwenianiania uczciwości poprzez implementację algorytmów osiągania konsensusu
    \item W systemie można planować zadania do wykonania
\end{itemize}
\vspace{0.3\baselineskip}
Podsumowując stworzyliśmy działający system z w pełni działającym graficznym interfejsem użytkownika, który pozwala na komunikację między użytkownikami i jak i tworzenie dowolnie dużych grup użytkowników, dla których odbywa się komunikacja grupowa. W systemie zapewniona jest poufność poprzez zastosowanie szyfrowania wiadmości, niezaprzeczalność przez podpisywanie cyfrowe bloków wiadomości i mechanizmy zapewniania uczciwości systemu przez algorytmu osiągania konsensusu.