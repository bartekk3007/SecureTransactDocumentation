\chapter{Wstęp i cel pracy}
\label{chap:wstep}

Ogólne wymagania dotyczące formatowania pracy dyplomowej wymienione zostały poniżej:
\begin{itemize}
\item	format arkusza: A4,
\item	orientacja papieru: pionowa,
\item	czcionka: Arial,
\item	wielkość czcionki podstawowej: 10 pkt,
\item	odstęp między wierszami (interlinia): 1,5 wiersza,
\item	marginesy (w odbiciu lustrzanym):
\begin{description}
\item[górny:] 2,5 cm,
\item[dolny:] 2,5 cm,
\item[wewnętrzny:] 3,5 cm,
\item[zewnętrzny:] 2,5 cm,
\end{description}
\item	tekst pracy powinien być wyrównany do obydwu marginesów (wyjustowany),
\item	każdy akapit należy rozpoczynać wcięciem 1,25 cm.
\end{itemize}
Praca przygotowywana jest do druku dwustronnego. Numeracja stron powinna być umieszczona w stopce dokumentu i wyśrodkowana. Na stronie tytułowej i zawierającej Oświadczenie numer strony nie może być widoczny. Od strony zawierającej Streszczenie (strona nr 3), a kończąc na ostatniej stronie pracy, numeracja powinna być ciągła i zapisana cyframi arabskimi, przy użyciu czcionki Arial 9 pkt.

Przykład prawidłowego sposobu podawania informacji w punktach (lista punktowana) pokazany jest powyżej. Każdy element listy należy rozpoczynać punktorem, a następujący za nim tekst małą literą. Wiersze wyliczania należy zakończyć przecinkiem albo średnikiem, a w przypadku ostatniego elementu wyliczania – kropką. 

Krój czcionek stosowanych w nagłówkach przedstawiono w tabeli \ref{tab:font_size}

\begin{table}[!ht]
\caption{Wielkość czcionki stosowanej w nagłówkach}
\label{tab:font_size}
\small
\begin{tabular}{|p{0.25\textwidth}|p{0.3\textwidth}|p{0.35\textwidth}|}
\hline
Poziom nagłówka &	Przykład &	Wielkość i styl czcionki\\ \hline
Nagłówek 1. stopnia	& \textbf{1. TYTUŁ ROZDZIAŁU} &	12 pkt, WERSALIKI, pogrubiona \\ \hline
Nagłówek 2. stopnia	& \textbf\textit{1.1. Podtytuł rozdziału} &	10 pkt, pogrubiona i kursywa \\ \hline
Nagłówek 3. stopnia	 & \textit{1.1.1. Punkt podrozdziału}	 & 10 pkt, kursywa \\ \hline
\end{tabular}
\end{table}
	
Nazwa tabeli jest umieszczona bezpośrednio nad nią, czcionka o wielkości 9 pkt, bez kropki na końcu, jak w zamieszczonym przykładzie. Odstępy zastosowane dla akapitu zawierającego opis tabeli są następujące:
\begin{enumerate}
\item	górny 6 pkt,
\item	dolny 0 pkt.
\end{enumerate}
Dane umieszczone w tabeli należy zapisać tak, jak w przykładowej tabeli \ref{tab:wielkosc_czcionki}, czyli 
z zastosowaniem czcionki o wielkości 9 pkt, wyrównując tekst do lewych krawędzi komórek tabeli. 
Numeracja tabel jest ciągła w ramach rozdziału. Numer porządkowy tabeli (w nazwie tabeli) poprzedzony jest słowem Tabela oraz numerem rozdziału i kropką (np. Tabela \ref{tab:wielkosc_czcionki} Wielkość …). Każda tabela musi być przywołana w tekście pracy, na przykład jak w zdaniu: „W tabeli \ref{tab:wielkosc_czcionki} zamieszczono …”. 

Jeśli zachodzi konieczność podziału tabeli między kolejne strony, należy pamiętać o powieleniu nagłówka tabeli na każdej ze stron

Pierwszy akapit pod tabelą rozpoczynamy odstępem górnym wynoszącym 12 pkt.
Na końcach wierszy nie pozostawiamy spójników i krótkich przyimków, takich jak: a, o, i, w, z, itd. Do ich przenoszenia zaleca się stosowanie tzw. twardej spacji (niełamliwej).
