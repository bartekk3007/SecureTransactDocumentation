\chapter{Interfejs graficzny użytkownika}
\label{chap:InterfejsGraficznyUżytkownika}
Aplikacja posiada rozbudowany graficzny interfejs użytkownika, który w sposób responsywny dostosowuje się do urządzeń desktopowych oraz mobilnych. Interfejs dostępny jest w dwóch trybach wizualnych: jasnym i ciemnym, oferując różnorodne efekty kolorystyczne oraz animacje. Dodatkowo aplikacja renderuje dynamicznie pojedynczy obiekt trójwymiarowy wraz z jego animacją, co wzbogaca wizualny aspekt interfejsu.

%---
\section{Projektowanie interfejsu graficznego}
\label{sec:ProjektowanieInterfejsuGraficznego}

\emph{Projektowanie interfejsu graficznego \ang{UI Design}}, to proces tworzenia wizualnego środowiska użytkownika przed jego implementacją programistyczną. W ramach naszego projektu inżynierskiego wykorzystałem profesjonalne narzędzie Figma \cite{Figma} do zaprojektowania interfejsu graficznego użytkownika, co pozwoliło usprawnić i ujednolicić dalsze prace związane z implementacją tego interfejsu w kodzie. Projekt graficzny stworzony w Figmie pełnił funkcję przewodnika, przy programowaniu interfejsu.

%---

\subsection{System Projektowania}
\label{sec:SystemProjektowania}

\emph{System projektowania \ang{Design system}}, to zestaw standaryzowanych danych i zasobów projektowych, które wspólnie definiują wygląd interfejsu graficznego i usprawniają proces projektowania. Celem Design systemu jest zapewnienie spójności wizualnej oraz ułatwienie współpracy między członkami zespołu. 
W ramach projektowania interfejsu graficznego, zacząłem od stworzenia prostego systemu projektowania, który ujednolicił paletę barw, definiując zestawy kolorów o różnych odcieniach jasności, rozmiary czcionek oraz odstępy między elementami. Te wartości stałe były konsekwentnie wykorzystywane zarówno w procesie projektowania, jak i w implementacji interfejsu graficznego, zapewniając spójność wizualną i funkcjonalną. 

%---

\subsection{Proces projektowania}
\label{sec:ProcesProjektowania}
Po zdefiniowaniu Systemu Projektowania rozpocząłem projektowanie stron aplikacji w dwóch rozdzielczościach: 1440x1024 dla urządzeń desktopowych oraz 375x667 dla urządzeń mobilnych. Dla każdej z tych rozdzielczości przygotowałem dwie wersje stron, różniące się układem elementów. Wszystkie projekty początkowo powstawały w kolorystyce \ang{Dark Mode}, a następnie były tworzone ponownie w jasnej wersji \ang{Light Mode}.

%---

\section{Implementacja Interfejsu Graficznego}
\label{sec:ImplementacjaInterfejsuGraficznego}
Język Python posiada wiele bibliotek do tworzenia interfejsów graficznych, nasz zespół zdecydował się na użycie frameworka QT w wersji 6 (Pyside6). Dzięki temu, że framework stanowi pakiet w języku Python nasz zespół mógł używać różnych IDE \ang{Integrated Development Environment}.
Framework ten posiada wiele narzędzi i komponentów umożliwiających tworzenie nie tylko interfejsów graficznych, od własnych struktur danych po moduły do pracy z bazami danych, jednakże w naszym projekcie głównie korzystaliśmy z języka Qt do Modelowania (QML) \ang{Qt Modeling Language} - deklaratywnego języka do tworzenia interfejsów w Frameworku QT. QML oraz QT jest wieloplatformowy, działa na urządzeniach desktopowych (Windows, Linux, MacOS) oraz urządzeniach mobilnych i wiele innych.
Dzięki QML możliwe było logiczne rozdzielenie backendu i frontendu, co pozwoliło na zapisanie definicji interfejsu graficznego w odrębnych plikach z rozszerzeniem .qml, ułatwiając tym samym zarządzanie kodem i jego strukturą.

\subsection{Uzasadnienie wyboru Frameworka QT}
\label{sec:UzasadnienieWyboruFrameworkaQT}
Przy wyborze technologii do tworzenia interfejsów graficznych w Pythonie mieliśmy do wyboru wiele innych bibliotek, na przykład Tkinter. 
Jednakże w porównaniu do Tkintera, interfejsy w tej bibliotece są przestarzałe wizualnie i mniej elastyczne. Biblioteka ta oferuje małe wsparcie dla chociażby efektów wizualnych, animacji czy responsywnego designu.

Framework QT oraz zawarty w nim QML mają wiele zalet, które nas przekonały do skorzystania z tej technologii:
\begin{enumerate}
    \item Qt jest wieloplatformowy, można tworzyć aplikacje na Windowsa, Linuxa, MacOSa i urządzenia mobilne
    \item Łatwo łączyć logikę backendu z frontendem
    \item QML umożliwia dodawanie animacji, renderowanie obiektów trójwymiarowych i tworzenie responsywnego interfejsu
    \item QT posiada ogromną dokumentację i ma szeroką społeczność użytkowników
    \item Oddzielenie frontendu (QML) od backendu (Python) sprzyja lepszej organizacji kodu
\end{enumerate}

\subsection{Podstawy QT Modeling Language} 
\label{sec:PodstawyQtModelingLanguage}
\emph{Qt Modeling Language} to język deklaratywny podobny do CSS i HTML, wchodzący w skład modułu QtQuick frameworka Qt. Aby zacząć używać QML'a, należy stworzyć plik .qml w którym możemy zawrzeć:

\begin{lstlisting}[language=QML, caption={Przykładowy kod QML}]
import QtQuick 2.0

Rectangle {
    id: kwadrat
    width: 100
    height: 100
    color: "red"
}
\end{lstlisting}
który tworzy kwadrat 100x100 w kolorze czerwonym, struktura kodu w QML jest hierarchiczna oznacza to, że wewnątrz kodu Rectangle moglibyśmy dodać kolejny obiekt, który byłby tworzony względem rodzica - Rectangle.

Dodatkowo QML umożliwia pisanie w języku Javascript w odpowiednich polach,
na przykład do kodu w kwadracie można dodać:

\begin{lstlisting}[language=QML, caption={Przykładowy Javascript}]
Component.onCompleted: {
    console.log(kwadrat.width);
}
\end{lstlisting}
który wypisze w konsoli szerokość kwadratu.


\subsection{Zaimplementowane istotne mechanizmy}
\label{sec:ZaimplementowaneIstotneMechanizmy}

\subsubsection{Struktura Interfejsu}
\label{sec:Struktura Interfejsu}
W naszym projekcie każdy osobny plik .qml stanowi osobną stronę interfejsu lub
definicję niestandardowego komponentu. Niestandardowe komponenty są zawarte
w osobnych folderach: \begin{lstlisting}
"app_style", "gui_components", "small_gui_components"
\end{lstlisting}

Omówię najpierw aspekt niestandardowych komponentów. W QMLu można tworzyć własne klasy komponentów.

\begin{lstlisting}[language=QML, caption={Przykładowa klasa w MyRectangle.qml}]
MyRectangle {
    id: rectangle
    color: "blue"
    width: 100
    height: 100
}
\end{lstlisting}
Zaimportowanie tego w innym pliku (Page1.qml) umożliwi stworzenie
wielu instancji klasy MyRectangle:

\begin{lstlisting}[language=QML, caption={Przykładowe użycie klasy MyRectangle}]
MyRectangle {} //niebieski kwadrat 100x100
MyRectangle {color: "red"} //czerwony kwadrat 100x100
\end{lstlisting}
Pozwala to na uniknięcie redundancji w kodzie, gdy trzeba tworzyć
wiele obiektów, różniących się, tylko niektórymi atrybutami.

W naszym projekcie jest to wykorzystywane przy tworzeniu list, na przykład
użytkowników albo własnych przycisków, których w interfejsie jest pełno i dzięki definiowaniu ich w jednej klasie, wyglądają podobnie i upraszczają kod.

Reszta plików \texttt{.qml} stanowi osobne strony interfejsu. 
Tworząc komponent StackView oraz używając:
\begin{lstlisting}[language=QML, caption={Ładowanie nowej strony}]
stackView.push("nazwa_pliku.qml");
\end{lstlisting}
można w bardzo prosty sposób załadować nową stronę.
Taki podział (strona - plik) sprawia, że struktura interfejsu jest bardziej
uporządkowana oraz logiczna i upraszcza pracę z kodem.

\subsubsection{Implementacja Systemu Projektowania}
\label{sec:ImplementacjaSystemuProjektowania}
System Projektowania z części Projektowania Interfejsu Graficznego zakłada użycie stałej palety kolorów, odstępów oraz wielkości czcionek.

Nasz projekt implementuje to w folderze \texttt{app\_style} w trzech plikach QML:

\begin{enumerate}
    \item ColorPalette.qml - Definiujący zestaw stałych kolorów
    \item FontStyle.qml - Definiujący czcionkę oraz różne rozmiary
    \item SpacingObjects.qml - Definiujący wartości odstępów między elementami w interfejsie
\end{enumerate}

Każdy z nich zawiera analogiczną logikę, więc omówię na przykładzie klasy \texttt{ColorPalette}:

W kodzie QML definiujemy stałe zmienne przy użyciu \texttt{property}

\begin{lstlisting}[language=QML, caption={ColorPalette.qml}]
import QtQuick 2.15

QtObject {
    readonly property color primary50: "#EBFFE5"
    readonly property color primary100: "#D7FFCC"
    readonly property color primary200: "#AFFF99"
    readonly property color primary300: "#87FF66"
    readonly property color primary400: "#5FFF33"
    readonly property color primary500: "#37FF00"
    readonly property color primary600: "#2CCC00"
    readonly property color primary700: "#219900"
    readonly property color primary800: "#166600"
    readonly property color primary900: "#0B3300"

    readonly property color secondary50: "#FEE5FF"
    readonly property color secondary100: "#FDCCFF"
    ...
    ...
\end{lstlisting}

Następnie w bardzo prosty sposób możemy używać tych stałych wartości w każdym miejscu w kodzie:
\begin{lstlisting}[language=QML, caption={Przykładowe użycie palety kolorów}]
import "../app_style"

//User (Peer) List Class Blueprint
Rectangle {
ColorPalette { id: colorPalette }
FontStyle { id: fontStyle }
SpacingObjects { id: spacingObjects }

    property string list_color: settings.light_mode ? colorPalette.background50 : colorPalette.background800
    ...
    ...
\end{lstlisting}  

\subsubsection{Komunikacja z Backendem}
\label{sec:KomunikacjaZBackendem}
Komunikacja z backendem w Frameworku QT odbywa się za pomocą sygnałów i slotów. Ten sam mechanizm istnieje również, gdy używamy QT wraz z językiem C++, jednakże syntax jest inny. 

Sygnał to obiekt, który jak nazwa wskazuje ma za zadanie wysłanie sygnału,
dokonuje się tego przy pomocy metody .emit().

Slot to specjalna funkcja, którą należy połączyć z wybranym, uprzednio stworzonym sygnałem.

Po wysłaniu sygnału zostaje wywołany ten podłączony Slot. Dodatkowo można przesyłać w metodzie .emit() argumenty do Slota.

Jeżeli chcemy osiągnać komunikację między backendem (plikami .py) oraz frontendem (.qml) to logicznie:

\begin{enumerate}
    \item Definiujemy sygnał w klasie w Pythonie, który potem należy podłączyć do zdefiniowanej w QML-u za pomocą JavaScript funkcji (slota) i następnie wywołać \texttt{.emit()} z poziomu Pythona dla tego sygnału. To umożliwi wywołanie funkcji zawartej w QML za pomocą kodu w Pythonie.
    \item Możemy też pójść w drugą stronę – jeśli chcemy wywołać coś z backendu przy użyciu kodu QML, wtedy należy stworzyć slot w klasie w Pythonie, gdzie kod QML staje się sygnałem.
\end{enumerate}

Istotnym aspektem jest to, że samo przekazanie wartości do frontendu jest operacją prostą, można tego dokonać jednorazowo przy inicjalnym ładowaniu interfejsu, o czym za chwilę, jednakże mechanizm sygnałów i slotów umożliwia dynamiczne przekazywanie informacji między frontendem i backendem.

Na przykład jeśli dodamy kogoś do znajomych, to w backendzie w funkcji która odpowiada za dodanie użytkownika do znajomych możemy wywołać .emit() i dynamicznie obsłużyć tą zmianę we Frontendzie.

Teraz pokazując kod z naszego projektu omówię na przykładzie jak wykorzystaliśmy ten mechanizm:

Przykładowo jeśli dynamicznie obsłużyć zmianę znajomych w naszym programie, to atrybut self.peers w klasie User() w Backendzie zawiera tablicę innych obiektów typu User, które chcemy przekazać do Frontendu i wypisać na ekranie.

Tworzymy sygnał w klasie User:
\begin{lstlisting}[language=Python, caption={Sygnał User}]
class User(QObject):
    peersChanged = Signal()
\end{lstlisting}    

Definiujemy tablicę peers jako Property:
\begin{lstlisting}[language=Python, caption={Peers Property}]
@Property("QVariantList", notify=peersChanged)
    def peers(self):
        return self._peers
\end{lstlisting}

W main.py przed inicjalnym załadowaniem interfejsu możemy przekazać
uprzednio stworzony pojedyńczy obiekt User, gdzie u nas jest to obecny użytkownik programu. Teraz w QML możemy używać metod i atrybutów tej klasy dla tego obiektu.

\begin{lstlisting}[language=Python, caption={Przekazanie inicjalnego obiektu}]
engine.rootContext().setContextProperty("user", user)
\end{lstlisting}

Teraz w klasie FriendList, po załadowaniu komponentu należy podłączyć sygnał peersChanged z funkcją javascript, która przetwarza dane znajomych z obiektu user.

\begin{lstlisting}[language=Python, caption={Podłączenie sygnału do slota w QML}]
Component.onCompleted: {
        user.peersChanged.connect(updateUserModel);
}
\end{lstlisting}
Teraz każdy peersChanged.emit() w pythonie spowoduje wywołanie funkcji updateUserModel, która u nas w projekcie, wygląda tak:

\begin{lstlisting}[language=Python, caption={Slot w QML obsługujący wyświetlanie zmienionej listy znajomych}]
function updateUserModel() {
    userModel.clear();

    // Iterate over peers array passed from Python
    for (let i = 0; i < user.peers.length; i++) {
        var activeColor = user.peers[i].active > 0 ? colorPalette.primary500 : colorPalette.destructive400

        var isInGroup = false;
        var isSelected = false;

        var host = user.peers[i].host;
        var port = user.peers[i].port;

        for (let j = 0; j < user.group.length; j++) {
            if (host === user.group[j].host && port === user.group[j].port) {
                isInGroup = true;
                break;
            }
        }

        userModel.append({
            nickname: user.peers[i].nickname,
            host: host,
            port: port,
            active: user.peers[i].active,
            isInGroup: isInGroup,
            isSelected: isSelected,
            activeColor: activeColor
        });
    }
}
\end{lstlisting}
Dla uproszczenia można założyć, że userModel to po prostu lista znajomych.
W ten sposób możemy dynamicznie aktualizować wyświetlane dane w Frontendzie za każdym razem, gdy lista znajomych (self.peers) ulegnie zmianie.

Na przykład funkcja odpowiadająca za usunięcie znajomego, która jest Slotem - wywoływana jest z poziomu Frontendu po kliknięciu przycisku, a Backend informuje Frontend o usunięciu znajomego za pomocą .emit(), a ten następnie przetwarza tą zmianę i wyświetla na ekranie:
\begin{lstlisting}[language=Python, caption={Przykładowe użycie sygnału dla listy znajomych}]
@Slot(str, int)
def removeFromPeers(self, host, port):
    for peer in self.peers:
        if peer.host == host and peer.port == port:
            self.peers.remove(peer)
            self.peersChanged.emit()
            break
\end{lstlisting}

\subsubsection{Renderowanie Obiektu Trójwymiarowego}
\label{sec:RenderowanieObiektuTrójwymiarowego}
Nasza aplikacja renderuje jeden obiekt trójwymiarowy, istotne jest to, że korzystamy z dynamicznego renderera obiektów 3D zamiast na przykład prostego .gifa.
QT umożliwia renderowanie obiektów 3D w QMLu, przy użyciu modułu Qt3D, który w naszym projekcie umożliwia narysowanie trójwymiarowego kształtu i jego animacje (obrót do okoła osi Y)

Przedstawię kod odpowiedzialny za to i następnie go omówię:
\begin{lstlisting}[language=QML, caption={Renderowanie trójwymiarowego bitcoina}]
View3D {
    id: view3D
    Layout.alignment: Qt.AlignHCenter
    anchors.fill: parent
    environment: sceneEnvironment

    SceneEnvironment {
        id: sceneEnvironment
        antialiasingQuality: SceneEnvironment.High
        antialiasingMode: SceneEnvironment.MSAA
    }

    Node {
        id: scene
        DirectionalLight {
            id: directionalLight
        }

        PerspectiveCamera {
            id: sceneCamera
            x: 0
            y: 0
            z: 500
        }

        Bitcoin {
            id: bitcoin

            x: -scale.x * 0.5
            y: -scale.y * 2
            z: -scale.z * 1

            property var scaleFactor: Math.sqrt(Math.min(root.width, root.height)) * 5

            scale: Qt.vector3d(scaleFactor, scaleFactor, scaleFactor)
        }
    }

    NumberAnimation {
        target: bitcoin
        property: "eulerRotation.y"
        loops: Animation.Infinite
        running: true
        from: 360
        to: 0
        duration: 10000
    }
}
\end{lstlisting}

TODO
...
...

% responsywność interfejsu i skalowanie komponentów
\subsubsection{Responsywność interfejsu}
\label{sec:ResponsywnośćInterfejsu}
TODO

\subsection{Pomoce naukowe związane z QT i QML}
\label{sec:PomoceNaukoweQML}
Przy pracy z Frameworkiem QT i QML korzystaliśmy z ogromnej oficjalnej dokumentacji: 
Dokumentacja QT \cite{DokumentacjaQT}
Dokumentacja QML \cite{DokumentacjaQML}
Dokumentacja QT for Python \cite{DokumentacjaQtForPython}.

Dodatkowo niektórzy członkowie zespołu korzystali z IDE przeznaczonego do pracy w QT: Qt Creator, który ma wbudowaną tą dokumentację i pod ręką na przykład można zaznaczyć obiekt w kodzie i kliknąć "f2", co sprawi, że otworzy się osobne okno w programie z dokumentacją tego obiektu. Z tego również czasem korzystaliśmy.

Dodatkowo korzystaliśmy z poradnika na youtubie:
Poradnik \cite{PoradnikQMLYoutube}

\subsection{Strategie rozwiązywania problemów w QML}
\label{sec:StrategieRozwiązywaniaProblemówWQML}
Podczas pracy w QML napotykaliśmy wiele różnych ciekawych problemów, które z czasem nauczyliśmy się rozwiązywać co raz to bardziej efektywnie.
TODO ...
