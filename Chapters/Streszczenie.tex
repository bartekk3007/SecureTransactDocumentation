\chapter*{Streszczenie}
\thispagestyle{plain}

Celem pracy było stworzenie systemu wspomagającego planowanie i realizację zadań, a także umożliwiającym komunikację tekstową między użytkownikami. Część wspomagająca planowanie i realizację zadań została wytworzona w oparciu o metodykę Getting Things Done (Załatwianie Spraw) autorstwa Davida Allena. 

Głównym założeniem aplikacji była decentralizacja, więc komunikacja została zrealizowana w architekturze rozproszonej wykorzystując sieć peer-to-peer (równy z równym). Dane są przechowywane z użyciem technologii łańcucha bloków \ang{blockchain}. Konsensus jest osiągany przy użyciu algorytmu dowodu stawki \ang{proof of stake}.

Te rozwiązania zwiększają bezpieczeństwo aplikacji, ponieważ w dużym stopniu utrudniają podsłuchanie danych, podszywanie się lub manipulację istniejących danych. Zmniejsza też ryzyko utraty danych bo jest więcej ich nośników. Dane przesyłane są protokołem HTTP. Są też szyfrowane i elektronicznie podpisywane co jeszcze bardziej zwiększa bezpieczeństwo. 

Interfejs aplikacji został napisany w platformie programostycznej QT, a logika biznesowa w języku Python.
\\\\
Słowa kluczowe: Blockchain, decentralizacja, planowanie, bezpieczeństwo
\\\\
Dziedzina nauki i techniki, zgodnie z wymogami OECD: Nauki inżynieryjne i techniczne - Elektrotechnika, elektronika, inżynieria informatyczna
