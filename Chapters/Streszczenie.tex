\chapter*{Streszczenie}
\thispagestyle{plain}

Celem pracy było stworzenie systemu wspomagającego planowanie i realizację zadań, a także umożliwiającym komunikację tekstową między użytkownikami. Część wspomagająca planowanie i realizację zadań została wytworzona w oparciu o metodykę Getting Things Done autorstwa Davida Allena. 

Głównym założeniem aplikacji była decentralizacja, więc komunikacja została zrealizowana w architekturze rozproszonej wykorzystując sieć Peer to Peer. Dane są przechowywane z użyciem technologii blockchain. Konsensus jest osiągany przy użyciu algorytmu proof of stake.

Te rozwiązania zwiększają bezpieczeństwo aplikacji, ponieważ w dużym stopniu utrudniają podsłuchanie danych, podszywanie się lub manipulację istniejących danych. Zmniejsza też ryzyko utraty danych bo jest więcej ich nośników. Dane przesyłane są protokołem HTTP. Są też szyfrowane i elektronicznie podpisywane co jeszcze bardziej zwiększa bezpieczeństwo. 

Interfejs aplikacji został napisany w frameworku QT, a logika biznesowa w języku Python.
\\\\
Słowa kluczowe: Blockchain, decentralizacja, planowanie, bezpieczeństwo
\\\\
Dziedzina nauki i techniki, zgodnie z wymogami OECD: Nauki inżynieryjne i techniczne, Sprzęt komputerowy i architektura komputerów
